\documentclass{standalone}


\begin{document}
	En este capítulo seguimos \cite{alberto:2023}.
	\noindent Para cualquier conjunto $V$, su conjunto potencia es denotado por $\mathcal{P}(V)$. La carnalidad de $V$ es $\# V$
	
	\begin{definition}\label{def:simplicial_complex}
		Un \emph{complejo simplicial} sobre un conjunto $V$ es un conjunto finito $K\subseteq\mathcal{P}(V)\setminus\{\emptyset\}$ cerrado bajo subconjuntos.
	\end{definition}
	
	Formalmente, la definición de arriba corresponde a la de los \emph{complejos simpliciales abstractos y finitos} \cite[Definition 2.1]{kozlov:2008:combinatorial:alg:topo}; dado que no estudiaremos otro tipo de complejos simpliciales omitimos los otros adjetivos. 
	
	\begin{remark}\label{rem:empty_complexes}
		En la Definición~\ref{def:simplicial_complex} tenemos dos complejos simpliciales que son importantes. El primero es el complejo simplicial \emph{vano} (\emph{void} en inglés), es decir $\emptyset\subseteq\mathcal{P}(V)$. El segundo es el complejo simplicial \emph{vacío} (\emph{empty} en inglés): $\{\emptyset\}$ \cite[Remark 2.3]{kozlov:2008:combinatorial:alg:topo}. 
	\end{remark}
	
	\section{Notación y nomenclatura estándar.}
	\noindent Sea $K$ un complejo simplicial. Cada elemento de $K$ se llama \emph{simplejo}\footnote{También es común en español, quizás lo es más, utilizar el término \emph{cara}. No lo adopto porque viene del uso geométrico de esa palabra: \textit{las caras del poliedro.} Aunque es verdad que los complejos simpliciales pueden considerarse objetos geométricos, su combinatoria es más natural. Véanse \cite{kozlov:2008:combinatorial:alg:topo}, \cite{may:1967:simplicial}} (\emph{simplex} y plural \emph{simplices} en inglés). La \emph{dimensión} de $\sigma$ es $\dim(\sigma)=\#\sigma-1$. Si $\dim(\sigma)=k$ decimos que $\sigma$ es un \emph{$k$-simplejo}. Un \emph{vértice} de un complejo simplicial es un $0$-simplejo; El conjunto de  $n$-simplejos de $K$ será denotado por $K_{n}$. 
	
	La dimensión de $K$ es $\dim(K)=\max\{\dim(\sigma)\mid\sigma\in K\}$. Una \emph{faceta} (o \emph{careta} si se usa cara) de un complejo simplicial es un simplejo máximal con respecto a la contención; es decir, $\sigma$ es faceta si $\sigma\subseteq\tau$, entonces $\tau=\sigma$. Diremos que un complejo simplicial es \emph{puro} wsiempre que todas sus facetas tengan la misma dimensión.
	
	Los complejos simpliciales son objectos combinatorios. Usualmente se representan gráficamente como sigue. Un punto representa un vértice, las aristas,  $1$-simplejos; los triángulos representan $2$-simplejos, etc. En Figure~\ref{fig:eg_simplex} we found examples of simplices and in Figure~\ref{fig:eg_simplicial_complex} we present examples of simplicial complexes. We remark that the hollow triangle in Figure~\ref{fig:eg_non_pure}  is not a simplex. Also, it is worth to notice that from the picture only it is not clear whether the tetrahedron in Figure~\ref{fig:3_simplex} is hollow or not. For these reason we only draw simplicial complexes for illustrative purposes.
	
	\begin{figure}[h]
		\centering
		\begin{subfigure}{.3\textwidth}
			\centering
			\subimport{images/}{eg_1_simplex}
			\caption{Un $1$-simplejo.}
			\label{fig:1_simplex}
		\end{subfigure}
		\begin{subfigure}{.3\textwidth}
			\centering
			\subimport{images/}{eg_2_simplex}
			\caption{Un $2$-simplejo.}
			\label{fig:2_simplex}
		\end{subfigure}
		\begin{subfigure}{.3\textwidth}
			\centering
			\subimport{images/}{eg_3_simplex}
			\caption{Un $3$-simplejo.}
			\label{fig:3_simplex}
		\end{subfigure}
		\caption{Representación gráfica de alguno simplejos. el tetraedro en \subref{fig:3_simplex} es un sólido $3$-dimensional.}
		\label{fig:eg_simplex}
	\end{figure}
	\begin{figure}[h]
		\centering
		\begin{subfigure}{.4\textwidth}
			\centering
			\subimport{images/}{eg_octahedral_sphere}
			\caption{A pure simplicial complex.}
			\label{fig:eg_octahedral_sphere}
		\end{subfigure}
		\begin{subfigure}{.4\textwidth}
			\centering
			\subimport{images/}{eg_non_pure_simplicial_complex}
			\caption{A non-pure simplicial complex.}
			\label{fig:eg_non_pure}
		\end{subfigure}
		\caption{Examples of simplicial complexes. All triangles in \subref{fig:eg_octahedral_sphere} are filled but the simplicial complex itself is not a $3$-dimensional solid.}
		\label{fig:eg_simplicial_complex}
	\end{figure}
\end{document}