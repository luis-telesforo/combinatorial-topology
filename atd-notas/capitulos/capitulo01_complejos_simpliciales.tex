\documentclass{standalone}


\begin{document}
	\noindent Para cualquier conjunto $V$, su conjunto potencia es denotado por $\mathcal{P}(V)$.
	
	\begin{definition}\label{def:simplicial_complex}
		Un \emph{complejo simplicial} sobre un conjunto $V$ es un conjunto finito $\Delta\subseteq\mathcal{P}(V)\setminus\{\emptyset\}$ cerrado bajo subconjuntos.
	\end{definition}
	
	Formalmente, la definición de arriba corresponde a la de los \emph{complejos simpliciales abstractos y finitos} \cite[Definition 2.1]{kozlov:2008:combinatorial:alg:topo}; dado que no estudiaremos otro tipo de complejos simpliciales omitimos los otros adjetivos. 
	
	\begin{remark}\label{rem:empty_complexes}
		En la Definición~\ref{def:simplicial_complex} tenemos dos complejos simpliciales que son importantes. El primero es el complejo simplicial \emph{vano} (\emph{void} en inglés), es decir $\emptyset\subseteq\mathcal{P}(V)$. El segundo es el complejo simplicial \emph{vacío} (\emph{empty} en inglés): $\{\emptyset\}$ \cite[Remark 2.3]{kozlov:2008:combinatorial:alg:topo}. 
	\end{remark}
\end{document}